\chapter{Conclusiones}

\bigskip

El \textit{Team Orienteering Problem} combina la decisión de qué clientes seleccionar con la decisión de cómo planificar la ruta. Al ser TOP un problema reconocido como modelo de muchas aplicaciones reales, se han generado varios trabajos que lo encaran. Incluso algunos pocos con algoritmos genéticos pero no encontré ninguno que implemente un BRKGA. Mi contribución al problema TOP consiste en generar una implementación que utilice como base de su construcción de soluciones al algoritmo BRKGA, mejorando las soluciones con búsquedas locales y analizar que tan efectivo es tal combinación de metaheurísticas.

\bigskip

Los resultados finales obtenidos son muy buenos, con al menos un 70\% de los resultados llegaron a la mejor solución conocida de la instancia testeada. El resto obtuvo un $i_{eMax}$ en el intervalo [0.97, 0.99], salvo algunos pocos que quedaron en el intervalo [0.94, 0.96]. Los resultados del BRKGA puro no fueron lo suficientemente buenos para instancias grandes del problema, llegando a tener un $i_{eMax}$ aproximado de $0.50$. Quizá esto se deba a como funciona el \textit{crossover} en TOP, las soluciones hijas terminan siendo muy diferentes de sus padres. Si ese fuera el caso, el BRKGA solo puede llegar a buenas soluciones con la ayuda de otras metaheurísticas como es el caso de mi desarrollo.

\bigskip

Uno de los problemas del BRKGA para TOP es que su secuencia de alelos no es utilizada de forma completa cuando se lo decodifica ya que parte del problema es que no todos los clientes pueden ser visitados. Por lo tanto, asignar todos los clientes a algún vehículo siempre generaría una solución no factible o la instancia del problema no sería del TOP. Este problema de matching entre alelos y clientes visitados quizá puede ser resuelto modificando lo que representa un gen. 

\bigskip

Al iniciar el desarrollo, se esperaba obtener buenos resultados del BRKGA puro en base a las características de los algoritmos genéticos tales como la supervivencia del mas apto, la herencia de las características del padre de elite, una población de mejora monotónica, etc. Sin embargo los resultados durante el desarrollo dejaron en claro que el BRKGA puro no obtuvo buenos resultados y que los resultados finales dependieron demasiado de las búsquedas locales. De todos modos este trabajo deja algunos aportes e ideas de como resolver ciertos problemas cuando se implementa un BRKGA. Tales como:

\bigskip

\begin{itemize}
	\item Se implementaron dos decodificadores, uno que genera una solución lo mas rápido posible y el otro que intenta generar mejores soluciones con una estrategia golosa. Se generaron diversas pruebas sobre estos decodificadores, se compararon y se analizaron sus resultados. 
	\item Se ideo e implementó un codificador de soluciones de modo que al combinar el BRKGA con búsquedas locales, el método de \textit{crossover} siga siendo independiente del problema que resuelve y logrando que los descendientes hereden genes que se corresponden a la aptitud de la solución de la cual es hija. 
	\item Se desarrollaron dos método para poder decidir equivalencia entre soluciones. Uno de los métodos aunque no asegura unicidad de soluciones dentro de una población, reduce considerablemente los repetidos y no requiere conocimiento del problema. Se realizaron pruebas de eficiencia entre ambos métodos.
	\item Se implementaron las búsquedas locales clásicas, pero ademas se hicieron variantes de algunas de forma que varíen la vecindad que exploran y se plantearon alternativas cuando el orden de complejidad era muy alto.
	\item A lo largo de toda la implementación el objetivo de llegar a la mejor solución posible se mantuvo sin dejar de lado la optimización temporal algorítmica. Como fue el caso de calcular el centro de gravedad de una ruta de modo de seleccionar primero los clientes mas cercanos a la ruta al intentar insertarlos pero solo calcularlo en el momento que es necesario y mantenerlo de un modo eficiente. También en el caso de la búsqueda local de reemplazo de múltiples clientes a la vez. Esta búsqueda tiene un orden de complejidad alto pero en la practica es muy bueno, por lo tanto se hicieron pruebas para verificar que en la practica no era tan costoso y se realizaron pruebas para medir cuanto aportaba a la solución final.
\end{itemize}


\section{Trabajos Futuros}

Sería útil tener una herramienta para visualizar las soluciones en un plano cartesiano, pudiendo ver rápidamente qué clientes se quedaron sin ser visitados y así poder idear alternativas para que los clientes cercanos sean incluidos. También para poder ver la similitud entre un individuo y los individuos descendientes, a modo de tener una idea clara de qué tan parecidos son. De todos modos, para un análisis más preciso de la correlación entre padres e hijos, sería más eficiente idear una función que analizando las rutas de ambas soluciones genere un índice de parentesco. 

\bigskip

El BRKGA puro para TOP necesita mejoras, los resultados obtenidos utilizando solo el BRKGA están lejos de ser competitivos. Si continuara el desarrollo del BRKGA sin búsquedas locales exploraría cambios en el decodificador y en el método de \textit{crossover}. 

\bigskip

En el decodificador buscaría alguna manera de que su secuencia de alelos se use de forma completa, es decir que todo alelo impacte en la formación de la solución. Para entender esto tomar como ejemplo el decodificador simple, una instancia con 10 clientes y digamos que la solución generada a partir de su secuencia de alelos visita a 6 de los 10 clientes. Por lo tanto, los últimos 4 alelos de la secuencia no impactan en el resultado final. Es decir, estos 4 alelos podrían cambiar de posición entre si y la solución resultante sería la misma. Quizá se podría implementar de tal forma, que los clientes se distribuyan uniformemente entre todos los vehículos y luego con un proceso de limpieza se convierta la solución en una factible. Otra alternativa sería implementar sectores preestablecidos asignados a un determinado vehículo, basados en cercanía ó el centro de gravedad del sector. Con ese índice de parentesco podríamos determinar cuan familiares son los descendientes y si hay diferencia entre realizar el \textit{crossover} o generar soluciones mutantes.

\bigskip

El segundo punto por el que intentaría mejorar los resultados del BRKGA es modificando el algoritmo de apareamiento. Quizá, el individuo resultante deba heredar rutas enteras. Entonces el individuo descendiente herede dos rutas de un padre y la tercer ruta del otro. Finalmente, con algún proceso de limpieza se muevan los clientes que se visitan de forma repetida y se incluyen otros. En este contexto, la cantidad de alelos que tendría una solución estaría dictaminado por la cantidad de vehículos. Esto podría representar un problema ya que existen muchos menos vehículos que clientes, generando baja diversidad de soluciones, es decir explorando muy poco el dominio de soluciones posibles.

\bigskip

Como vimos en los resultados parciales durante el desarrollo de la implementación, el algoritmo BRKGA puro no tuvo éxito resolviendo TOP. En instancias grandes el $i_eAvg$ apenas llegó a obtener valores de 0.50 aproximadamente. Es bastante probable que esto se deba a una baja correlación entre padres e hijos. Es decir, que las soluciones generadas a partir del \textit{crossover} sean soluciones muy distintas a sus padres. Si este fuera el caso, no habría mucha diferencia entre generar una solución al azar y el método de \textit{crossover} que se implementó. Por lo tanto, como potencial trabajo futuro se podría hacer una investigación exhaustiva para determinar si existe combinación de método de \textit{crossover} y decodificador tal que el BRKGA puro llegue a valores de $i_eAvg$ de 0.80 o más. Para esta investigación habría que plantear diversos combinaciones de métodos de \textit{crossover} y de decodificadores soluciones. Luego formular distintos índices de similitud entre soluciones para medir parentesco. Finalmente realizar un test estadístico para determinar si existe correlación entre las implementaciones con un alto índice de parentesco y buenos resultados finales. En otras palabras tales trabajos futuros deberían poder responder las siguientes preguntas:

\begin{itemize}
	\item ¿Existe combinación de método de \textit{crossover} y de decodificador de soluciones tal que el BRKGA puro obtenga un  $i_eAvg > 0.80$?
	\item ¿Existe correlación entre buenos resultados del BRKGA y una alta similitud entre padres e hijos?
\end{itemize}

\bigskip

Sobre trabajos futuros relacionados con las búsquedas locales, se podría implementar la búsqueda \textit{Move}, para mover un cliente visitado de una ruta hacia otra, acumulando mayor distancia libre en una sola ruta. También se podría implementar alguna heurística local tabú de modo de salir de mínimos locales. De todos modos, no continuaría por las búsquedas locales ya que es un tema que esta muy desarrollado.

\bigskip

Los resultados finales fueron muy buenos, en el caso de continuar trabajando en mi desarrollo haría foco en las ideas sobre cambio del método de \textit{crossover} y en el decodificador del BRKGA.








