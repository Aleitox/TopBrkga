\chapter{Modelo Matemático}

La formulación para el TOP donde el punto de inicio y fin son el mismo, ha sido presentada por Tang H. y Miller-Hooks E. \cite{TangMillerHooks}. Tal formulación puede ser extendida al caso donde el punto de inicio y fin pueden ser diferentes. A continuación presentaré la formulación matemática extendida del TOP propuesta por Ke L., Archetti C. y Feng Z. \cite{KeArchettiFeng}.

\bigskip

Sea un grafo completo $G=(V,E)$ donde $V=\{1,...,n\}$ es el conjunto de vértices y $E=\{(i,j)|i,j \in V\}$ es el conjunto de ejes. Cada vértice $i$ en $V$ tiene un beneficio $r_i$. El punto de inicio es el vértice 1, el punto de fin es el vértice $n$ y $r_1=r_n=0$. Todo eje $(i,j)$ en $E$, tiene un costo no negativo $c_{ij}$ asociado, donde $c_{ij}$ es la distancia entre $i$ y $j$. El TOP consiste en encontrar $m$ caminos que comiencen en el vértice 1 y terminen en el vértice $n$ de forma tal que el beneficio total de los vértices visitados sea maximizado. Cada vértice debe ser visitado a lo sumo una sola vez. Para cada vehículo, el tiempo total para visitar los vértices no puede superar un limite pre-especificado $d_{max}$. En el presente modelo matemático se asume que hay una proporcionalidad directa entre la distancia recorrida de un vehículo y el tiempo consumido por el vehículo. Por lo tanto no hay diferencia en considerar $d_{max}$ como una distancia o un tiempo. Para evitar conflictos el valor es considerado como el valor de distancia máxima.

\bigskip

Sea $y_{ik} = 1 (i = 1,...,m)$ si $\exists j \in V$ tal que el eje $(i,j)$ es visitado por el vehículo $k$, sino $y_{ik} = 0$. Sea $x_{ijk} = 1$ $(1 \leq i < j \leq n; 1 \leq k \leq m)$ si el eje $(i,j)$ es visitado por el vehículo $k$, sino $x_{ijk} = 0$. Sea $U$ un subconjunto de $V$. Luego TOP puede ser descrito de la siguiente manera:

\bigskip

\begin{equation}
max \sum_{i=2}^{n-1} \sum_{k=1}^{m} r_i y_{ik}
\end{equation}

sujeto a

\begin{equation} \label{eq:modelo2}
\sum_{j=2}^{n} \sum_{k=1}^{m} x_{1jk} = \sum_{i=1}^{n-1} \sum_{k=1}^{m} x_{ink} = m
\end{equation}

\begin{equation} \label{eq:modelo3}
\sum_{i<j} x_{ijk} + \sum_{i>j} x_{jik} = 2y_{jk} \quad (i =2,...,n-1) \quad (k = 1,...,m)
\end{equation}

\begin{equation} \label{eq:modelo4}
\sum_{k=1}^{m} y_{ik} \leq 1 \quad (i =2,...,n-1)
\end{equation}

\begin{equation} \label{eq:modelo5}
\sum_{i=1}^{n-1} \sum_{j>i} c_{ij}x_{ijk} \leq d_{max} \quad (k=1,...,m)
\end{equation}

\begin{equation} \label{eq:modelo6}
\sum_{i,j\in U \ i<j} x_{ijk} \leq |U|-1 \quad (U \subset V \setminus \{1,n\} ; 3 \leq |U| \leq n-2; k=1,...,m)
\end{equation}

\begin{equation} \label{eq:modelo7}
x_{ijk} \in \{0,1\} \quad (1 \leq i < j \leq n; k=1,...,m)
\end{equation}

\begin{equation} \label{eq:modelo8}
y_{1k} = y_{nk} = 1, \quad y_{ik} \in \{0,1\} \quad (i = 2,...,n-1; k=1,...,m)
\end{equation}

\bigskip

La restricción \ref{eq:modelo2} asegura que todo vehículo comienza en el vértice 1 y termina en el vértice $n$. La restricción \ref{eq:modelo3} asegura la conectividad de cada camino. La restricción \ref{eq:modelo4} asegura que cada vértice (excepto el 1 y el $n$) debe ser visitado a los sumo una vez. La restricción \ref{eq:modelo5} describe la limitación de distancia. La restricción \ref{eq:modelo6} asegura que no hay ciclos. La restricción \ref{eq:modelo7} y \ref{eq:modelo8} establecen que todas las variables son enteras.


