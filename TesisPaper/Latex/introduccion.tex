\chapter{Introducción}

La orientación \cite{ChaoGoldenWasil} es un deporte originario de Escandinavia jugado al aire libre normalmente en bosques o zonas montañosas. Con ayuda de un mapa y una brújula, un competidor comienza en un punto de control específico e intenta visitar tantos otros puntos de control como le sea posible dentro de un límite de tiempo prescrito y regresa a un punto de control especificado. Cada punto de control tiene una puntuación asociada, de modo que el objetivo de este deporte es maximizar la puntuación total. Un competidor que llegue al punto final después de que el tiempo haya expirado es descalificado. El competidor elegible con la puntuación más alta es declarado ganador. Dado que el tiempo es limitado, un competidor puede no ser capaz de visitar todos los puntos de control. Por lo tanto cada competidor debe seleccionar un subconjunto de puntos de control para visitar que maximizarán la puntuación total. Este problema se conoce como el \textit{Orienteering Problem} y se denota por OP.

\bigskip

El equipo de orientación extiende la versión de un solo competidor del deporte a un equipo formado por varios competidores (digamos 2, 3 o 4 miembros). Todos los competidores comienzan en el mismo punto y cada miembro del equipo intenta visitar tantos puntos de control como le sea posible dentro de un límite de tiempo prescrito, terminando en el punto final. Una vez que un miembro del equipo visita un punto y se le otorga la puntuación asociada, ningún otro miembro del equipo puede obtener una puntuación por visitar el mismo punto. Por lo tanto, cada miembro de un equipo tiene que seleccionar un subconjunto de puntos de control para visitar, de modo que dos miembros del mismo equipo no visiten el mismo punto de control, el límite de tiempo no sea violado y la puntuación total del equipo sea maximizada. Este problema se conoce como el \textit{Team Orienteering Problem} y lo denotan por TOP.

\bigskip

El OP es NP-Completo como demostraron Golden, Levy, y Vohra \cite{GoldenLevyVohra}, por lo que el TOP es al menos tan difícil ya que lo contiene. Es por este motivo que la mayoría de las propuestas para estos problemas se han centrado en proporcionar enfoques heurísticos.

\bigskip

El TOP ha sido reconocido como un modelo de muchas aplicaciones reales diferentes como por ejemplo:

\begin{enumerate}[i]

\item

El deporte de orientación de equipo explicado anteriormente (ver Chao et al \cite{ChaoGoldenWasil}). 

\item

Algunas aplicaciones de servicios de recogida o entrega que implican el uso de transportistas comunes y flotas privadas (ver Ballou y Chowdhury \cite{BallouChowdhury}). 

\item

La planificación de viajes turísticos donde existen varios puntos de interés que el turista quiere visitar. Cada uno de estos puntos de interés tienen un valor dado por el turista y un tiempo mínimo para poder visitarlo.

\item

El problema de entrega de combustible con múltiples vehículos de Golden, Levy y Vohra \cite{GoldenLevyVohra}. Una flota de camiones debe entregar combustible a una gran cantidad de clientes diariamente. Una característica clave de este problema es que el suministro de combustible del cliente debe mantenerse en un nivel adecuado en todo momento. Es decir, cada cliente tiene una capacidad de tanque conocida y se espera que su nivel de combustible permanezca por encima de un valor crítico preespecificado que puede denominarse punto de reabastecimiento. Las entregas siguen un sistema de empuje en el sentido de que están programados por la empresa en base a un pronóstico de los niveles de los tanques de los clientes. Los desabastecimientos son costosos y deben evitarse cuando sea posible.

\item

El reclutamiento de jugadores de fútbol americano universitario de Butt y Cavalier \cite{ButtCavalier}. Un método exitoso de reclutamiento utilizado en muchas pequeñas divisiones del \textit{National Collegiate Athletic Association} es visitar los campus de las escuelas secundarias y reunirse con los miembros superiores de los equipos de fútbol americano. A modo de maximizar su potencial para reclutar futuros jugadores, deben visitar tantas escuelas secundarias como sea posible dentro de un radio de 100 km del campus, sabiendo por experiencia previa que visitar todas las escuelas en esta área no es posible. Por lo tanto, deben visitar el mejor subconjunto de escuelas en el área.

\item

El enrutamiento de técnicos para atender a los clientes en ubicaciones geográficamente distribuidas. En este contexto, cada vehículo en el modelo TOP representa un solo técnico y hay a menudo una limitación en el número de horas que cada técnico puede programar para trabajar en un día dado. Por lo tanto, puede no ser posible incluir a todos los clientes que requieren servicio en los horarios de los técnicos para un día determinado. En su lugar, se seleccionará un subconjunto de los clientes. Las decisiones sobre qué clientes elegir para su inclusión en cada uno de los horarios de los técnicos de servicio pueden tener en cuenta la importancia del cliente o la urgencia de la tarea. Este requisito de selección de clientes también surge en muchas aplicaciones de enrutamiento en tiempo real.

\end{enumerate}

Para la generación y comparación de resultados se utilizaron instancias de test de Tsiligirides y de Chao \cite{IntancesChaoTsiligirides}. Las instancias de problemas de ambos autores comparten el mismo formato. 

\bigskip

\begin{minipage}{\textwidth}
Una instancia de TOP contiene:

\begin{itemize}
  \item $N$ vehículos de carga. Cada vehículo tiene una distancia máxima, llamada $d_{max}$, que puede recorrer. En esta implementación cada vehículo puede tener una distancia máxima diferente. De todos modos en las instancias de test utilizadas, los vehículos tienen el mismo $d_{max}$.
  \item $M$ clientes. Un cliente es un nodo con beneficio mayor a cero. Cada cliente tiene un set de coordenadas $X$ e $Y$ que representan su ubicación en un plano cartesiano.
  \item Un nodo de inicio y fin de ruta. Todos los vehículos inician y finalizan el recorrido en estos nodos. Ambos nodos tienen un beneficio de cero y tienen un set de coordenadas $X$ e $Y$.
\end{itemize}
\end{minipage}

\bigskip

\begin{minipage}{\textwidth}
También es importante mencionar que:

\begin{itemize}
	\item Todos los clientes en las instancias del benchmark tienen dos coordenadas y se utiliza la distancia euclidiana para medir distancias.
	\item Una solución es válida si:
	\begin{itemize}
		\item Para todo vehículo, la distancia de su ruta es menor o igual al $d_{max}$ del vehículo que realiza tal ruta.
		\item Ningún cliente pertenece a dos rutas distintas.
		\item Toda ruta parte del nodo de inicio y finaliza en el nodo de fin.
	\end{itemize}
	\item La función objetivo retorna la sumatoria de los beneficios de los clientes visitados.
\end{itemize}
\end{minipage}

\bigskip

En el capítulo 2 se encuentra la revisión bibliográfica, donde se sintetizaron las propuestas de los trabajos previos que resolvieron TOP.
En el capítulo 3 se presentará el modelo matemático de TOP que presentaron Tang y Miller-Hooks \cite{TangMillerHooks} .
En el capítulo 4 describiré los algoritmos genéticos en general, luego se explica en particular el RKGA y el BRKGA. Por último se comenta sobre el algoritmo decodificador que utiliza el BRKGA.
En el capítulo 5 se detalla en profundidad la implementación de toda la solución, comenzando por los decodificadores utilizados, su eficiencia, desventajas y comportamiento. Luego se explica mi implementación del algoritmo BRKGA, detallando las configuraciones testeadas, resultados parciales y problemas encontrados. Por último en este mismo capítulo se describen las búsquedas locales implementadas, el objetivo de cada una, los distintos ordenes en que se aplicaron las búsquedas locales y los resultados parciales sobre un subconjunto diverso del benchmark de instancias.
En el capítulo 6 muestro los resultados finales obtenidos sobre las instancias del benchmark de problemas.
Por último en el capítulo 7 comento sobre las conclusiones y trabajos futuros.


























