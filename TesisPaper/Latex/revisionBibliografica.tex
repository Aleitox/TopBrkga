

\chapter{Revisión Bibliográfica}

Existe una gran cantidad de aplicaciones que pueden ser modelas por TOP. Es por esto que la clase de problemas de enrutamiento de vehículos con ganancias es amplia. En el 2013, C. Archetti, M.G. Speranza, D. Vigo \cite{ArchettiSperanzaVigo} publicaron una revisión sobre esta clase de problemas. En este capítulo voy a referenciar varias de la publicaciones que referenciaron, y otras trabajos que encontré relacionados con TOP.

\bigskip

La primera heurística propuesta para el TOP es un algoritmo de construcción simple introducido en Butt y Cavalier \cite{ButtCavalier} y probado en pequeñas instancias de tamaño con hasta 15 nodos. En su heurística \textit{MaxImp}, se asignan pesos a cada par de nodos de modo que cuanto mayor es el peso, más beneficioso es no solo visitar esos dos nodos, sino visitarlos en el mismo recorrido. Su peso depende de cuán beneficioso sean los nodos y las sumas de las distancias de ir y volver por esos nodos. Fue introducido con el nombre \textit{Multiple Tour Maximum Collection Problem}.

\bigskip

Una heurística de construcción más sofisticada se da en Chao, Golden y Wasil (CGW) \cite{ChaoGoldenWasil}. Este es la primer publicación donde aparece el nombre TOP. Este nombre fue acuñado por Chao et al. para resaltar la conexión con el más ampliamente estudió caso de un solo vehículo (OP). La solución inicial se refina a través de movimientos de los clientes, los intercambios y varias estrategias de reinicio. En este trabajo mencionan que TOP puede ser modelado como un problema de optimización multinivel. En el primer nivel, se debe seleccionar un subconjunto de puntos para que el equipo visite. En el segundo nivel, se asignan puntos a cada miembro del equipo. En el tercer nivel, se construye un camino a través de los puntos asignados a cada miembro del equipo. El algoritmo resultante se prueba en un conjunto de 353 instancias de prueba con hasta 102 clientes y hasta 4 vehículos.

\bigskip

El primer algoritmo exacto para TOP es propuesto por Butt y Ryan \cite{ButtRyan}. Comienzan a partir de una formulación de partición configurada y su algoritmo hace un uso eficiente tanto de la generación de columnas como de la bifurcación de restricciones. El algoritmo es capaz de resolver instancias con hasta 100 clientes potenciales cuando las rutas incluyen solo unos pocos clientes cada uno. Más recientemente, Boussier et al. \cite{BoussierFeilletGendreau} presentaron un algoritmo de \textit{Branch and Price}. Gracias a diversos procedimientos de aceleración en el paso de generación de columnas, puede resolver instancias con hasta 100 clientes potenciales del gran conjunto de instancias de referencia propuestas en Chao et al. \cite{ChaoGoldenWasil}.

\bigskip

Luego, se aplicaron varias metaheurísticas al TOP, partiendo del algoritmo de \textit{Tabu Search} introducido en Tang y Miller-Hooks (TMH) \cite{TangMillerHooks}, que está incorporado en un \textit{Adaptive Memory Procedure} (AMP) que alterna entre vecindarios pequeños y grandes durante la búsqueda. La heurística de búsqueda tabú propuesta por TMH para el TOP se puede caracterizar en términos generales en tres pasos: inicialización, mejora de la solución y evaluación. Paso A, iniciación desde el AMP: dada la solución actual S determinada en el AMP, establece los parámetros tabu a una pequeña etapa del vecindario en la que solo se explorará una pequeña cantidad de soluciones de vecindario. Paso B, mejora: genera mediante procedimientos aleatorios y golosos una cantidad de soluciones de vecindario (validas e invalidas) a la solución actual en función de los parámetros tabú actuales. Note que en iteraciones selectas, la secuencia de cada una de estas soluciones de vecindario se mejora mediante procedimientos heurísticos. Paso C, evaluación: Se selecciona la mejor solución que no sea tabú entre los candidatos generados en el paso B (el estado tabu puede anularse si la mejor solución tabu es mejor que la mejor solución factible actual). Dependiendo del tamaño actual del vecindario y la calidad de la solución, se establece el parámetro del tamaño del vecindario en etapas grandes o pequeñas y regrese al Paso A o al B. Nótese que como señala Golden et al. \cite{GoldenLaporteTaillard}, el AMP funciona de forma similar a los algoritmos genéticos, con la excepción de que la descendencia (en AMP, las nuevas soluciones iniciales) se puede generar a partir de más de dos padres. Sus resultados de experimentos computacionales realizados sobre el mismo conjunto de problemas de Chao et al. muestran que la técnica propuesta produce consistentemente soluciones de alta calidad y supera a otras heurísticas publicadas hasta tal momento para el TOP.

\bigskip

Archetti et al. \cite{ArchettiHertzSperanza} proponen dos variantes de un algoritmo de un \textit{Tabu Search} generalizado y un algoritmo llamado \textit{Variable Neighborhood Search} (VNS). El VNS parte de una solución titular $s$, desde donde dan un salto a una solución $s'$. Se llama salto porque se hace dentro de un vecindario más grande que el vecindario utilizado para la búsqueda tabú. Luego aplican una búsqueda tabú en $s'$ para tratar de mejorarla. La solución resultante $s''$ se compara luego con $s$. Si se sigue una estrategia VNS, entonces $s''$ se convierte en el nuevo titular solo si $s''$ es mejor que $s$. En la estrategia de búsqueda tabú generalizada, se establece $s = s''$ incluso si $s''$ es peor que $s$. Este proceso se repite hasta que se cumplan algunos criterios de detención.

\bigskip

Ke et al. \cite{KeArchettiFeng} proponen un \textit{Ant colony Optimization} (ACO) que utiliza cuatro métodos diferentes para construir soluciones candidatas. ACO es una clase de metaheurísticas basadas en población. Utiliza una colonia de hormigas, que están guiadas por rastros de feromonas e información heurística, para construir soluciones de forma iterativa para un problema. El procedimiento principal se puede describir de la siguiente manera: una vez que se inicializan todos los rastros y parámetros de feromonas, las hormigas construyen soluciones iterativamente hasta que se alcanza un criterio de detención. El procedimiento iterativo principal consta de dos pasos. En el primer paso, cada hormiga construye una solución de acuerdo con la regla de transición. Entonces se puede adoptar un procedimiento de búsqueda local para mejorar una o más soluciones. En el segundo paso, los valores de las feromonas se actualizan de acuerdo con una regla de actualización de feromonas. Un punto clave en ACO es construir soluciones candidatas. Proponen cuatro métodos, los métodos secuencial, determinista-concurrente, aleatorio-concurrente y simultáneo.

\bigskip

Los autores Vansteenwegen et al. \cite{VansteenwegenSouffriauBergheOudheusden}, crearon un algoritmo compuesto donde primero construyen una solución y luego la mejoran con una combinación de heuristicas de búsqueda local. Tales heuristicas como \textit{Swap}, \textit{Replace}, \textit{Move}, \textit{Insert} y \textit{2-Opt}. Una vez que la solución es mejorada, si es la mejor encontrada hasta el momento la guardan. Luego tienen un método para encontrar nuevas soluciones partiendo de una solución, quitándole destinos a las rutas y así poder explorar distintas opciones.

\bigskip

Souffriau et al. \cite{SouffriauVansteenwegenBergheOudheusden} proponen una metaheuristica de 'Path Relinking' para resolver TOP. Su propuesta esta basada en un 'Greedy Randomised Adative Search Procedure' (GRASP), una metaheuristica introducida por Feo y Resende. A grandes rasgos su algoritmo consiste de una iteración que continua mientras no se exceda la máxima cantidad de iteraciones sin mejoras. El loop contiene cuatro fases. La primera de construcción de la solución, seguido por la fase de búsqueda local donde aplican \textit{Swap}, \textit{Replace}, \textit{Insert} y \textit{2-Opt}. La tercer fase es de enlace con los elites donde toma dos soluciones como argumentos y visita todas las soluciones intermedia para de ir de una solución hacia la otra. La última fase del loop es actualizar el conjunto de soluciones elite. Finalmente cuando corta el loop retorna la mejor solución encontrada.

\bigskip

Bouly et al. \cite{BoulyDangMoukrim} idearon un algoritmo memético para resolver TOP. Los algoritmos meméticos (MA) son una combinación de un algoritmo evolutivo y tecnicas de busqueda local. Se dice que una codificación es indirecta si se necesita un procedimiento de decodificación para extraer soluciones de los cromosomas. En su trabajo usan una codificación indirecta simple que denotan como un recorrido gigante, y un procedimiento de división óptima como el proceso de decodificación. La división óptima fue presentada por primera vez por Beasley (1983) y Ulusoy (1985), respectivamente, para los problemas Node Routing y el Arc Routing. El procedimiento de división que propusieron es específico del TOP.

\bigskip

Dang et al. \cite{DangGuibadjMoukrim}, este siendo el mejor actual en su clase, proponen un algoritmo memético basado en 'Particle Swarm Optimization'
(PSO). Su algoritmo PSOMA provee de soluciones de alta calidad para TOP. El algoritmo está relativamente cerca de MA propuesto en Bouly et al. \cite{BoulyDangMoukrim} y presenta los mismos componentes básicos, como la técnica de división de rutas, el inicializador de población y los barrios de búsqueda local. Sin embargo, el esquema global se ha modificado por una optimización de enjambre de partículas. La optimización de enjambre de partículas (PSO) es una de las técnicas de inteligencia de enjambre con la idea básica de simular la inteligencia colectiva y el comportamiento social de los animales salvajes.

\bigskip

 Ferreira et al. \cite{FerreiraQuintasOliveiraPereiraDias} implementan un algoritmo genético para resolver TOP. Su algoritmo consiste básicamente de tres componentes. El más elemental, que llaman cromosoma, representa un conjunto de vehículos y sus rutas. El segundo componente es su proeso de evolución, responsable de hacer el cruzamiento (crossover) y mutaciones dentro de una población. Su último componente vendría a ser el algoritmo responsable de controlar el proceso evolutivo, asegurandose que los cromosomas (soluciones) sean validas respecto de las restricciones de la instancia de TOP. En su proceso de cruzamiento se toman dos cromosomas y generan dos nuevos cromosomas utilizando aleatoriamente rutas de los cromosomas originales.


\bigskip

Esos fueron los trabajos encontrados en mi investigación sobre trabajos previos, hay algunos que implementen algoritmos geneticos pero ninguno que implemente un biased random key generation algorithim (BRKGA).









