%\begin{center}
%\large \bf \runtitulo
%\end{center}
%\vspace{1cm}
\chapter*{\tituloAbstractEs}

\noindent The team orienteering problem (TOP) is the generalization to the case of multiple tours of the Orienteering Problem, know also as Selective Traveling Salesman Problem (TSP). TOP involves finding a set of paths from the starting point to the ending point such that the total collected reward received from visiting a subset of locations is maximized and the length of each path is restricted by a pre-specified limit. In this thesis, a biased random key genetic algorithm (BRKGA) approach is proposed for the team orienteering problem. Also, In every population generation, the best N results are inhanced with a sequence of local search heuristics. Computational experiments are made on standard instances. Then, this results, were compared to the results obtained by Chao, Golden, and Wasil (CGW), Tang and Miller-Hooks (TMH) and Archetti, Hertz, Speranza (AHS). Though my results where very good and competite in most intances, in some they were not as good as mentioned previous works.

\bigskip

\noindent\textbf{Palabras claves:} Problema de orientación de equipo, Biased Random Key Genetic Algorithm, Routing Problem, Local Search Heuristic, Greedy Solution Construction.