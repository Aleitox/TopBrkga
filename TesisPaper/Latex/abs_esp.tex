
\chapter*{\tituloAbstractEs}

\noindent El problema de orientación de equipo (TOP) es la generalización al caso de múltiples recorridos del problema de orientación, también conocido como Problema del vendedor ambulante selectivo (TSP). TOP implica encontrar un conjunto de rutas desde el punto de inicio hasta el punto final, de modo que la recompensa total obtenida al visitar un subconjunto de ubicaciones se maximice y la longitud de cada ruta esté restringida por un límite preestablecido. En esta tesis, se propone un enfoque de algoritmo genético de clave aleatoria sesgada (BRKGA) para el problema de orientación de equipo. Además, en cada generación de población, la mejor solución no mejorada se mejora con una secuencia de heurística de búsqueda local. Los experimentos computacionales se realizan en instancias estándar. Luego, estos resultados se compararon con los resultados obtenidos por Chao, Golden y Wasil (CGW), Tang y Miller-Hooks (TMH) y Archetti, Hertz, Speranza (AHS). Aunque mis resultados fueron muy buenos y competitivos en la mayoría de los casos, en otros no fueron tan buenos como los trabajos mencionados anteriormente.

\bigskip

\noindent\textbf{Keywords:} Problema de Orientación de Equipo, Algoritmo Genético de Clave Aleatoria Sesgada, Problema de Enrutamiento, Heurística de Búsqueda Local, Construcción de Soluciones Golosas.