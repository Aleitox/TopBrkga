
\chapter*{\tituloAbstractEs}

\noindent En el \textit{Orienteering Problem} (OP) \cite{Tsiligirides}, se da un conjunto de nodos, cada uno con un beneficio determinado. El objetivo es determinar una ruta, limitada en su longitud, que visite a algunos nodos maximizando la suma de los beneficios obtenidos. El OP se puede formular de la siguiente manera: dado $n$ nodos en el plano euclidiano cada uno con un beneficio, donde $beneficio(nodo_i) \geq 0$ y $beneficio(nodo_1) = beneficio(nodo_n) = 0$, se debe encontrar una ruta de beneficio máximo a través de estos nodos, iniciando en el $nodo_1$ y finalizando en el $nodo_n$, de longitud no mayor que $d_{max}$. En el artículo \textit{Heuristic Methods Applied to Orienteering} de Tsiligirides \cite{Tsiligirides} se llama a este problema como \textit{Generalized Traveling Salesman Problem} (GTSP).

\bigskip

El \textit{Team Orienteering Problem} (TOP) \cite{ChaoGoldenWasil} es la generalización al caso de múltiples rutas del \textit{Orienteering Problem}. Resolver el TOP implica encontrar un conjunto de rutas desde el nodo de inicio hasta el nodo final de forma tal que se maximice la sumatoria de los beneficios recolectados, la distancia de todas las rutas no supere a $d_{max}$ y ningún nodo sea visitado más de una vez. El OP pertenece a la clase problemas NP-Completo ya que contiene al problema \textit{Traveling Salesman Problem} como caso especial (ver Garey y Johnson \cite{GareyJohnson}). De la misma manera, el TOP pertenece a la clase de problemas NP-Completo porque contiene al OP como un caso especial donde solo hay una ruta. Resolver el TOP requiere determinar el orden en que se visitan los nodos y además, seleccionar qué subconjunto de nodos a visitar, ya que no necesariamente se visitan todos los nodos.

\bigskip

En este trabajo propongo una combinación del \textit{Biased Random Key Genetic Algorithm} (BRKGA) \cite{Bean} y de búsquedas locales para resolver el TOP. El BRKGA es una clase de algoritmos genéticos cuya población inicial es generada utilizando un decodificador que convierte un conjunto de vectores de números enteros aleatorios, en un conjunto de soluciones válidas del problema. El BRKGA es una variante del \textit{Random Key Genetic Algorithm} (RKGA). Estos algoritmos se diferencian en el proceso de apareamiento (\textit{crossover}), mientras que en el RKGA los padres son elegidos al azar entre todos los individuos de la población, en el BRKGA uno de los padres siempre pertenece al subconjunto de los mejores individuos de la población y este padre tiene mayor probabilidad de trasmitir sus genes al individuo resultante del proceso de apareamiento.

\bigskip

En mi algoritmo, en cada nueva generación, la mejor solución se mejora con algunas búsquedas locales. Dada una solución $s$, un algoritmo de búsqueda local básicamente busca mejores soluciones en la vecindad de $s$. La solución $s'$ en la vecindad de $s$, es mejor que $s$ si el beneficio total recolectado por $s'$ es mayor al de $s$ o si sus beneficios recolectados son iguales y la distancia recorrida por las rutas de $s'$ es menor a la de $s$. En este trabajo implementé los algoritmos de búsqueda local: \textit{Insert}, \textit{Swap}, \textit{2-Opt}, \textit{Simple Replace} y \textit{Mutiple Replace}.

\bigskip

Los experimentos computacionales los realicé en instancias estándar de la literatura. Las instancias se dividen siete conjuntos. Los primeros tres conjuntos de instancias son los de Tsiligirides \cite{Tsiligirides} y los siguientes cuatro conjuntos son los de Chao et al. \cite{ChaoGoldenWasil}. Todas las instancias pueden encontrarse en \cite{IntancesChaoTsiligirides}. Mis resultados fueron comparados con los resultados obtenidos por los siguientes autores: Chao, Golden y Wasil \cite{ChaoGoldenWasil} (CGW), Tang y Miller-Hooks \cite{TangMillerHooks} (TMH), Archetti, Hertz, Speranza \cite{ArchettiHertzSperanza} (AHS), Ke, Archetti y Feng \cite{KeArchettiFeng} (KAF) y Bouly, Dang y Moukrim \cite{BoulyDangMoukrim} (BDM). 

\bigskip

Los resultados de mi algoritmo son muy buenos dado que para el 70\% de las instancias mi implementación obtuvo la mejor solución conocida y para el 30\% restante obtuvo valores competitivos con los trabajos previos mencionados.

\bigskip

\noindent\textbf{Palabras clave:} Team Orienteering Problem, Biased Random Key Genetic Algorithm, Routing Problem, Problema de Enrutamiento, Búsqueda Local, Decodificador.