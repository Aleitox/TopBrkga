\chapter{Concluciones}

\bigskip

El \textit{Team Orienteering Problem} combina la decisión de qué clientes seleccionar con la decisión de cómo planificar la ruta. Al ser TOP un problema reconocido como modelo de muchas aplicaciones reales, se han generado varios trabajos que encaran el mismo. Incluso algunos pocos con algoritmos genéticos pero no encontré ninguno que implemente un BRKGA. Mi contribución al problema TOP consiste en generar una implementación que utilice como base de su construcción de soluciones al algoritmo BRKGA y analizar que tan efectivo puede ser el mismo. La implementación final utiliza el algoritmo BRKGA y además, unas búsquedas locales para mejorar algunos individuos selectos de cada generación. La efectividad de la implementación final terminó dependiendo fuertemente de las heurísticas de búsqueda local.

\bigskip

Los resultados obtenidos son muy buenos, con al menos un tercio de los resultados llegaron a la solución optima de la instancia testeada. El resto obtuvo un $i_{eMax}$ en el intervalo [0.95, 0.99], salvo por algunas excepciones. Ahora, si considero el resultado obtenido por el BRKGA puro, los resultados no son lo suficientemente buenos para instancias grandes del problema, llegando a tener un $i_{eMax}$ aproximado de $0.5$. Quizá por como funciona el \textit{crossover} en TOP, las soluciones hijas terminan siendo muy diferentes de su padres. Si ese fuera el caso, el BRKGA solo puede llegar a buenas soluciones con la ayuda de otras metaheurísticas como es en este caso.

\bigskip

Considero que uno de los problemas del BRKGA para TOP es que su secuencia de alelos no es utilizada completamente ya que parte del problema es que no todos los clientes pueden ser visitados. Asignar todos los clientes a algún vehículo siempre generaría una solución no factible o la instancia del problema no sería de TOP. Este problema de matching entre alelos y clientes visitados quizá puede ser resuelto modificando lo que representa un gen. Es decir, haciendo un decodificador nuevo.

\section{Trabajos Futuros}

Sería útil tener una herramienta para visualizar las soluciones en un plano cartesiano, pudiendo ver rápidamente que clientes se quedaron sin ser visitados y así poder idear alternativas para que los clientes cercanos sean incluidos. También para poder ver la similitud entre individuos y sus individuos progenitores, a modo de tener una idea clara de que tan parecidos son. De todos modos, para un análisis mas preciso de tal correlación, sería mas eficiente idear una función que analizando las rutas de ambas soluciones genere un índice de parentesco.

\bigskip

El BRKGA puro necesita mejoras, no estoy del todo conforme con los resultados obtenidos utilizando solo el BRKGA. Si continuara mi desarrollo del BRKGA sin búsquedas locales exploraría cambios en el decodificador y en el método de \textit{crossover}. 

\bigskip

En el decodificador buscaría alguna manera de que su secuencia de alelos se use de forma completa, es decir que todo alelo impacte en la formación de la solución. Para entender esto tomar como ejemplo el decodificador goloso y sea una instancia con 10 clientes y una solución generada a partir de su secuencia de alelos que visita a 6 de los 10 clientes, luego los últimos 4 alelos de la secuencia no impactan en el resultado final. Es decir, estos 4 alelos los podría cambiar de posición entre si y la solución seria la misma. Quizá podría implementarse de tal forma, que los clientes se distribuyan uniformemente entre todos los vehículos y luego con un proceso de limpieza se convierta la solución en una factible. Sino, que de alguna manera existan pre establecido sectores asignados a un solo vehículo, basados en cercanía ó el centro de gravedad del sector. Esto es por donde continuaría la investigación y el desarrollo.

\bigskip

El segundo punto por el que intentaría mejorar los resultados del BRKGA es modificando el algoritmo de apareamiento. Quizá cada alelo represente una ruta de un vehículo. Luego el individuo descendiente herede dos rutas de un padre y la tercer ruta del otro, finalmente con algún proceso de limpieza se muevan los clientes que se visitan de forma repetida y se incluyen otros. En este contexto, la cantidad de alelos que tendría una solución estaría dictaminado por la cantidad de vehículos. Esto podría representar un problema ya que existen muchos menos vehículos que clientes, generando baja diversidad de soluciones, es decir explorando muy poco el dominio de soluciones posibles.

\bigskip

Sobre trabajos futuros relacionados con las búsquedas locales, se podría implementar la búsqueda \textit{Move}, para mover un cliente visitado de una ruta hacia otra, acumulando mayor distancia libre en una sola ruta. También se podría \textit{Replace} que actualmente cambia a un cliente visitado por otro no visitado si incrementa el beneficio total. La optimización sería buscar un \textit{Replace} que no necesariamente sea de uno por uno. Podría ser el caso que hayan dos cliente no visitados cuyo beneficio total supere el de un cliente visitado, pero el beneficio de cada uno por separado, sea menor que el visitado. En este escenario el \textit{Replace} actual no efectuá el cambio. También se podría implementar alguna heurística local tabú de modo de salir de mínimos locales.

\bigskip

El objetivo de este trabajo era analizar el rendimiento del algoritmo BRKGA para TOP, es por esto que de hacer trabajos futuros sobre el tema haría foco en las ideas sobre cambio del método de \textit{crossover} y en el decodificador del BRKGA.








