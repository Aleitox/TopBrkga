\chapter{Resultados}

En este capítulo mostraré los resultados finales obtenidos. Dentro de todos los papers de trabajos previos que encontré que incluyeran resultados que utilizaran el benchmark de intancias de Chao y los de Tsiligirides, seleccioné el de Archetti et al. \cite{ArchettiHertzSperanza}. En su paper presentan varias soluciones, algunas utilizando un tabu search y otras utilizando una busqueda de vecindario variable. Luego las comparan con los resultados de Chao et al. Chao \cite{ChaoGoldenWasil} y de Tang et al. \cite{TangMillerHooks}. Dentro de las soluciones propuestas de Archetti et al., seleccione los resultados de su 'Slow VNS Feasible' ya que tenía los mejores resultados al correrlos con el benchmark de instancias. 

\bigskip

La configuración final utilizada para el BRKGA fue:

\begin{itemize}
  \item \textbf{MinIterations}: 200
  \item \textbf{MinNoChanges}: 50
  \item \textbf{PopulationSize}: 150
  \item \textbf{ElitePercentage}: 0.3 
  \item \textbf{MutantPercentage}: 0.1
  \item \textbf{EliteGenChance}: 70 
  \item \textbf{Heuristics}: Swap, Insert, 2-Opt, Replace, Swap, 2-Opt, Replace
  \item \textbf{ApplyHeuristicsToTop}: 2
  \item \textbf{DecoderType}: Simple
\end{itemize}

Además agregué un último paso una vez que corta el algoritmo, antes de entregar la mejor solución encontrada se le aplica una secuencia larga de heuristicas locales distinta a la aplicada a los mejores individuos de cada generación.

\bigskip

Los resultados fueron generados corriendo la implementación en una laptop hp con las siguientes especificaciones:

\begin{itemize}
  \item Procesador: Intel Core i7 5500u
  \item Memoria: DDR3 12 GBytes
  \item Graphics: Intel HD Graphics 5500
  \item Sistema Operativo: Windows 10 64-bit Home
  \item Ide: Visual Studio Enterprice 2015
  \item Lenguaje: C\# .Net Framework 4.5
\end{itemize}

\bigskip

A continuación los mejores resultados de 199 instancias del bechmark para mi algoritmo (LK), el de Archetti, Hertz y Speranza(AHS), Tang y Miller-Hooks(TMH) y Chao, Golden, y Wasil (CGW). En el caso de mi algoritmo, para cada instancia lo ejecute diez veces y me quede con el mejor resultado obtenido.