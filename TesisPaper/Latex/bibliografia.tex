\renewcommand\bibname{References}
\begin{thebibliography}{9}

\bibitem{referenceId} 
Autores
\textit{Paper title}. 
Editora y Fecha
 
\bibitem{otroReferenceId} 
Name: Site Title,
\\\texttt{http://google.com}

\bibitem{ArchettiSperanzaVigo} 
[Ref1] C. Archetti, M.G. Speranza, D. Vigo.
\textit{Vehicle Routing Problems with Profits.} 
Department of Economics and Management, University of Brescia, Italy 2013

\bibitem{ArchettiHertzSperanza} 
[8] C. Archetti, A. Hertz, and M.G. Speranza.
\textit{Metaheuristics for the team orienteering problem.} 
Journal of Heuristics, 13:49–76, 2007.

\bibitem{referenceId} 
[19] H. Bouly, D.-C. Dang, and A. Moukrim.
\textit{A memetic algorithm for the team orienteering problem.}
4OR, 8:49–70, 2010.

\bibitem{ButtCavalier} 
[21] S.E. Butt and T.M. Cavalier.
\textit{A heuristic for the multiple tour maximum collection problem.}
Computers and Operations Research, 21:101–111, 1994.

\bibitem{ChaoGoldenWasil} 
[24] I-M. Chao, B.L. Golden, and E.A. Wasil.
\textit{The team orienteering problem.}
European Journal of Operational Research, 88:464–474, 1996.

\bibitem{Dang} 
[26] D.-C. Dang, R.N. Guibadj, and A. Moukrim.
\textit{A PSO-based memetic algorithm for the team orienteering problem.}
In C. Di Chio, A. Brabazon, G. Di Caro, R. Drechsler, M. Farooq, J. Grahl, G. Greenfield, C. Prins, J. Romero, G. Squillero, E. Tarantino, A. Tettamanzi, N. Urquhart, and A. Uyar, editors, Applications of Evolutionary Computation, Lecture Notes in Computer Science, pages 471–480. Springer, Berlin, 2011.

\bibitem{goldenlevyvohra} 
[43] B.L. Golden, L. Levy, and R. Vohra.
\textit{The orienteering problem.}
Naval Research Logistics, 34:307–318, 1987.

\bibitem{KeArchettiFeng} 
[50] L. Ke, C. Archetti, and Z. Feng.
\textit{Ants can solve the team orienteering problem.}
Computers and Industrial Engineering, 54:648–665, 2008.

\bibitem{Souffriau} 
[68] W. Souffriau, P. Vansteenwegen, G. Vanden Berghe, and D. Van Oudheusden.
\textit{A path relinking approach for the team orienteering problem.}
Computers and Operations Research, 37:1853–1859, 2010.

\bibitem{TangMillerHooks} 
[70] H. Tang and E. Miller-Hooks.
\textit{A tabu search heuristic for the team orienteering problem.}
Computers and Operations Research, 32:1379–1407, 2005.

\bibitem{Vansteenwegen} 
[77] P. Vansteenwegen, W. Souffriau, G. Vanden Berghe, and D. Van Oudheusden.
\textit{A guided local search metaheuristic for the team orienteering problem.}
European Journal of Operational Research, 196:118–127, 2009.

\bibitem{referenceId} 
[80] Goldberg, D.
\textit{Genetic algorithms in search, optimization and machine learning.}
1st Ed., Addison-Wesley, Massachusetts, 1989.

\bibitem{referenceId} 
[81] Bean, J.C.
\textit{Genetic algorithms and random keys for sequencing and optimization.}
ORSA J. Comput. 6, 154–160 (1994)

\bibitem{referenceId} 
Villiam M. Spears , Kenneth A. De Jong
\textit{On the virtues of parameterized uniform crossover.}
1991




\bibitem{latexcompanion} 
Michel Goossens, Frank Mittelbach, and Alexander Samarin. 
\textit{The \LaTeX\ Companion}. 
Addison-Wesley, Reading, Massachusetts, 1993.
 
\bibitem{einstein} 
Albert Einstein. 
\textit{Zur Elektrodynamik bewegter K{\"o}rper}. (German) 
[\textit{On the electrodynamics of moving bodies}]. 
Annalen der Physik, 322(10):891–921, 1905.
 
\bibitem{otroReferenceId} 
Name: Site Title,
\\\texttt{http://google.com}
\end{thebibliography}

% [Ref1] C. Archetti, M.G. Speranza, D. Vigo. Vehicle Routing Problems with Profits. Department of Economics and Management, University of Brescia, Italy 2013
% [8] C. Archetti, A. Hertz, and M.G. Speranza. Metaheuristics for the team orienteering problem. Journal of Heuristics, 13:49–76, 2007.
% [19] H. Bouly, D.-C. Dang, and A. Moukrim. A memetic algorithm for the team orienteering problem. 4OR, 8:49–70, 2010.
% [21] S.E. Butt and T.M. Cavalier. A heuristic for the multiple tour maximum collection problem. Computers and Operations Research, 21:101–111, 1994.
% [24] I-M. Chao, B.L. Golden, and E.A. Wasil. The team orienteering problem. European Journal of Operational Research, 88:464–474, 1996.
% [26] D.-C. Dang, R.N. Guibadj, and A. Moukrim. A PSO-based memetic algorithm for the team orienteering problem. In C. Di Chio, A. Brabazon, G. Di Caro, R. Drechsler, M. Farooq, J. Grahl, G. Greenfield, C. Prins, J. Romero, G. Squillero, E. Tarantino, A. Tettamanzi, N. Urquhart, and A. Uyar, editors, Applications of Evolutionary Computation, Lecture Notes in Computer Science, pages 471–480. Springer, Berlin, 2011.
% [43] B.L. Golden, L. Levy, and R. Vohra. The orienteering problem. Naval Research Logistics, 34:307–318, 1987.
% [50] L. Ke, C. Archetti, and Z. Feng. Ants can solve the team orienteering problem. Computers and Industrial Engineering, 54:648–665, 2008.
% [68] W. Souffriau, P. Vansteenwegen, G. Vanden Berghe, and D. Van Oudheusden. A path relinking approach for the team orienteering problem. Computers and Operations Research, 37:1853–1859, 2010.
% [70] H. Tang and E. Miller-Hooks. A tabu search heuristic for the team orienteering problem. Computers and Operations Research, 32:1379–1407, 2005.
% [77] P. Vansteenwegen, W. Souffriau, G. Vanden Berghe, and D. Van Oudheusden. A guided local search metaheuristic for the team orienteering problem. European Journal of Operational Research, 196:118–127, 2009.
% [80] Goldberg, D., "Genetic algorithms in search, optimization and machine learning" 1st Ed., Addison-Wesley, Massachusetts, 1989.
% [81] Bean, J.C.: Genetic algorithms and random keys for sequencing and optimization. ORSA J. Comput. 6, 154–160 (1994)
% [82] Villiam M. Spears , Kenneth A. De Jong: On the virtues of parameterized uniform crossover (1991)
